\documentclass[12pt]{article}

\begin{document}

\title{Hypothesis}
\author{Shiro Takagi}
\date{2021/4/12}
\maketitle

\section{question}
Q: How to make an intelligent system that learns to manipulate symbols?

\section{survey}

\subsection{cognitive science}
Chomsky says that the power to taming syntax is innate property of 
human brain (universal grammar) \cite{Chomsky02}. 
Ibbotson says that ``the complexity of language emerges 
not as a result of a language-specific instinct 
but through the interaction of cognition and use" (usage-based theory) 
\cite{Ibbotson13}. According to the usage-based theory, linguistic 
structure develops by 1. categorization, 2. chunking, 3. rich memory, 
4. analogy, and 5. cross-modal association \cite{Bybee10,Ibbotson13}.
Children use a limited number of reliable short frames.


``
\textit{Overall it seems there is good evidence to support the usage-based 
prediction that language structure emerges in ontogeny out of 
experience (viz. use) and when a child uses core usage-based cognitive 
processes – categorization, analogy, form-meaning mapping, chunking, 
exemplar/item-based representations – to find and use communicatively 
meaningful units.} \cite{Ibbotson13}
"

The meaning of symbols is established by convention \cite{Santoro21,Taniguchi18,Mcclelland20}. 


\subsection{transformers}
Transformer learns syntactic information \cite{Reif19,Hewitt19,Goldberg19,Tenney19}.

\subsection{comparative study}
Watson et al. claims that ``\textit{nonadjacent dependency processing, a crucial cognitive facilitator 
of language, is an ancestral trait that evolved at least ~40 million years before language itself}'' \cite{Watson20}. 
Wilson et al. explains that sufficient cues play crucial role for human nonadjacent dependency learning \cite{Wilson20}.
Okanoya and Merker propose the hypothesis that human language is established through string-context 
mutual segmentation: ``\textit{song strings and behavioral contexts are mutually segmented
during social interactions}'' \cite{Okanoya07}. 

\subsection{Neo-Vygotskian Theory}
\textit{The key novelties in human evolution were all, in one way or another, adaptations for an especially 
cooperative, indeed hypercooperative, way of life} \cite{Tomasello19}. Tomasello states that there 
are two human-unique capabilities: joint intentionality and collective intentionality. He says that 
Human developed preparation for these abilities and children first acquire joint intention and then 
collective intention in their developmental stages.

\section{Discussion}
\subsection{transformers}
If transformers really capture syntax, \textbf{how it develops the syntactic representation during the pre-training? }
Previous studies seem to find that pre-trained transformer have syntactic representation but how to do that remains 
to be answered. If I can single out the cause of the syntax emergence, I may be able to model a guiding principle 
for an intelligent agent to learn syntax.

\subsection{syntax}
It might be plausible that infants first identify phrases in sentences. To identify phrases, it might be necessary 
that the phrase is used in multiple sentences. By observing the subset of sentences in multiple sentences repeatedly, 
infants could identify which subset is the the phrase. However, a phrase less likely to appear many times in multiple contexts. 
Thus, I could hypothesize that infants first understand too common and too often appearing phrase. Then, they 
understand that theres is the concept ``phrase" in their society. Finally, they could start to generalize their knowledge 
and to do try and error to manipulate phrase order. In sum,\textbf{ I hypothesize that artificial intelligence should 
follow the following path for language acquisition: phrase identification - phrase order arrangement - phrase manipulation. }
If this is the case, I should create an environment where identifying key phrase will give reword to the agent. 
This view is similar to that in \cite{Okanoya07} which I explained above.

\subsection{keyword detection}
Dual-coding memory may be a key because visual information is pseudo label there \cite{Hill21}.
Even if no instructor exists, the agent can learns the concept him/herself.

\subsection{collective intention}
A distinct feature of human intelligence is its mastery of languages. On the other hand, 
Tomasello argues that humans differ from other animals in terms of collective intentionality.
If this is the case, \textbf{a natural hypothesis from this is that human develop language skills 
thanks to their collective intentionality.}

\subsection{Action and Symbolic Manipulation}
Human understand a notion of ``research'' and know that it includes ``science'', for example. 
Or, I can use a more abstract notion of ``object'' and apply an ``operation'' on it. Human beings 
seem to excel at these symbolic manipulation. I believe that memory is ``used'' for these operations. 

I support the view that the semantics of neural representation is designed through the culture 
the agent is in \cite{Santoro21}. Repeated activations of neurons constructs an abstract concept 
and the symbols are attached to these concepts and segments the neural representation space. 
Thus, the abstract concept is formed through energy minimization and semantics is grounded to these 
abstract through social interactions. I hypothesize that human attach the symbol because it 
improves the predictability of internal and external states. In other words, I do so 
because it is useful. I think that the relation is also formed through this process.

\textbf{I think that the symbolic operation is a generalization 
of the physical action in the neural representation space.} When I move a cup, the cup will change 
its position after the action. In the similar vein, I act on a symbol and produce another symbol. 
Planning, making a hypothesis, proving a proposition, all of these mental actions can be 
regarded as like this. I find a book that presents a similar idea a bit \cite{Hawkins21}. 
I also think that symbolic operation occurs just because it is ``useful''.

Therefore, I believe that the temporal characteristics of the environment humans live in matters 
for symbolic manipulation.
% \section{Note}
% As Shannon pointed out, this kind of redundancy allows recovery of meaning 
% xvxn whxn thx sxgnxl xs nxxsy <-- Masked Language Model? \cite{Shannon51}
% Syntax representation development during pre-training is left for future study.
% should memory be multi-modal?

\subsection{Language and Localization}
A research proposes an interesting view on localization of human language function in our brain \cite{Macneilage09}. 
They say that ``\textit{our hypothesis holds that the left
hemisphere of the vertebrate brain was originally specialized for the control of well-established
patterns of behavior under ordinary and familiar circumstances. In contrast, the right hemisphere, 
the primary seat of emotional arousal, was at first specialized for detecting and responding to 
unexpected stimuli in the environment ...  In other words, the left hemisphere became the seat of self-motivated behavior, sometimes called top-down control. (We
stress that self-motivated behavior need not be
innate; in fact, it is often learned.) The right
hemisphere became the seat of environmentally
motivated behavior, or bottom-up control.}'' \cite{Macneilage09}. 
This view is consistent with 
\cite{Okanoya07}, where authors emphasize the importance of ``song'' and repeated phrases for 
the emergence of human language. An important implication from this literature is the 
connections among repeated action, top-down control, and language. 
\textbf{An intelligent agent may come to manipulate symbols after gaining top-down control through 
repeated actions.} 

They present another interesting observation. They note that ``\textit{In humans the right hemisphere
“takes in the whole scene,” attending to the global aspects of its environment 
rather than focusing on a limited number of features. That capacity gives it substantial advantages in analyzing
spatial relations. Memories stored by the right hemisphere tend to be organized and recalled as overall patterns 
rather than as a series of single items. In contrast, the left hemisphere tends to
focus on local aspects of its environment.}'' \textbf{If this is the case, I may assume that 
locality preference bias may be a good starting point for nurturing symbolic manipulations.}

They also hypothesize that ``\textit{to assess an incoming stimulus, an organism must carry out two kinds of analyses simultaneously. 
It must estimate the overall novelty of the stimulus and take decisive emergency action
if needed (right hemisphere). And it must determine whether the stimulus fits some familiar category, 
so as to make whatever well-established response, if any, is called for (left hemisphere).}'' 
\textbf{If I adopt this view, I may say that I can assume two kind of intrinsic motivations. 
One is the preference for the novelty, which is studied for a long time \cite{Schmidhuber10}. 
The other is the preference for categorization, which is not explored but can be important for 
higher cognitive functions.} They also say that ``\textit{Perhaps, then, those hemispheric specializations 
initially evolved because collectively they do a more efficient job of processing both kinds
of information at the same time than a brain without such specialized systems.}'' 



\bibliography{ref}
\bibliographystyle{plain}

\end{document}