\documentclass[12pt]{article}

\begin{document}

\title{Hypothesis}
\author{Shiro Takagi}
\date{2021/4/12}
\maketitle

\section{question}
Q: How to make an intelligent system that learns to manipulate symbols?

\section{survey}

\subsection{cognitive science}
Chomsky says that the power to taming syntax is innate property of 
human brain (universal grammar) \cite{Chomsky02}. 
Ibbotson says that ``the complexity of language emerges 
not as a result of a language-specific instinct 
but through the interaction of cognition and use" (usage-based theory) 
\cite{Ibbotson13}. According to the usage-based theory, linguistic 
structure develops by 1. categorization, 2. chunking, 3. rich memory, 
4. analogy, and 5. cross-modal association \cite{Bybee10,Ibbotson13}.
Children use a limited number of reliable short frames.


``
\textit{Overall it seems there is good evidence to support the usage-based 
prediction that language structure emerges in ontogeny out of 
experience (viz. use) and when a child uses core usage-based cognitive 
processes – categorization, analogy, form-meaning mapping, chunking, 
exemplar/item-based representations – to find and use communicatively 
meaningful units.} \cite{Ibbotson13}
"

The meaning of symbols is established by convention \cite{Santoro21,Taniguchi18,Mcclelland20}. 


\subsection{transformers}
Transformer learns syntactic information \cite{Reif19,Hewitt19,Goldberg19,Tenney19}.


\section{Note}
% As Shannon pointed out, this kind of redundancy allows recovery of meaning 
% xvxn whxn thx sxgnxl xs nxxsy <-- Masked Language Model? \cite{Shannon51}

\bibliography{ref}
\bibliographystyle{plain}

\end{document}