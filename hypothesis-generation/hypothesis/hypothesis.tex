\documentclass[12pt]{article}

\begin{document}

\title{Hypothesis}
\author{Shiro Takagi}
\date{2021/4/12}
\maketitle

\section{question}
Q: How to make an intelligent system that learns to manipulate symbols?

\section{survey}

\subsection{cognitive science}
Chomsky says that the power to taming syntax is innate property of 
human brain (universal grammar) \cite{Chomsky02}. 
Ibbotson says that ``the complexity of language emerges 
not as a result of a language-specific instinct 
but through the interaction of cognition and use" (usage-based theory) 
\cite{Ibbotson13}. According to the usage-based theory, linguistic 
structure develops by 1. categorization, 2. chunking, 3. rich memory, 
4. analogy, and 5. cross-modal association \cite{Bybee10,Ibbotson13}.
Children use a limited number of reliable short frames.


``
\textit{Overall it seems there is good evidence to support the usage-based 
prediction that language structure emerges in ontogeny out of 
experience (viz. use) and when a child uses core usage-based cognitive 
processes – categorization, analogy, form-meaning mapping, chunking, 
exemplar/item-based representations – to find and use communicatively 
meaningful units.} \cite{Ibbotson13}
"

The meaning of symbols is established by convention \cite{Santoro21,Taniguchi18,Mcclelland20}. 


\subsection{transformers}
Transformer learns syntactic information \cite{Reif19,Hewitt19,Goldberg19,Tenney19}.

\subsection{comparative study}
Watson et al. claims that ``\textit{nonadjacent dependency processing, a crucial cognitive facilitator 
of language, is an ancestral trait that evolved at least ~40 million years before language itself}'' \cite{Watson20}. 
Wilson et al. explains that sufficient cues play crucial role for human nonadjacent dependency learning \cite{Wilson20}.
Okanoya and Merker propose the hypothesis that human language is established through string-context 
mutual segmentation: ``\textit{song strings and behavioral contexts are mutually segmented
during social interactions}'' \cite{Okanoya07}. 

\section{Discussion}
\subsection{transformers}
If transformers really capture syntax, \textbf{how it develops the syntactic representation during the pre-training? }
Previous studies seem to find that pre-trained transformer have syntactic representation but how to do that remains 
to be answered. If we can single out the cause of the syntax emergence, we may be able to model a guiding principle 
for an intelligent agent to learn syntax.

\subsection{syntax}
It might be plausible that infants first identify phrases in sentences. To identify phrases, it might be necessary 
that the phrase is used in multiple sentences. By observing the subset of sentences in multiple sentences repeatedly, 
infants could identify which subset is the the phrase. However, a phrase less likely to appear many times in multiple contexts. 
Thus, we could hypothesize that infants first understand too common and too often appearing phrase. Then, they 
understand that theres is the concept ``phrase" in their society. Finally, they could start to generalize their knowledge 
and to do try and error to manipulate phrase order. In sum,\textbf{ we hypothesize that artificial intelligence should 
follow the following path for language acquisition: phrase identification - phrase order arrangement - phrase manipulation. }
If this is the case, we should create an environment where identifying key phrase will give reword to the agent. 

\subsection{keyword detection}
Dual-coding memory may be a key because visual information is pseudo label there \cite{Hill21}.
Even if no instructor exists, the agent can learns the concept him/herself. 


\section{Note}
% As Shannon pointed out, this kind of redundancy allows recovery of meaning 
% xvxn whxn thx sxgnxl xs nxxsy <-- Masked Language Model? \cite{Shannon51}
% Syntax representation development during pre-training is left for future study.
% should memory be multi-modal?

\bibliography{ref}
\bibliographystyle{plain}

\end{document}